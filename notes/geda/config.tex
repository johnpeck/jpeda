\section{gEDA configuration files}

\subsection{The \texttt{gafrc} file}
I use the \texttt{gafrc} file to tell \progname{gschem} where my
schematic symbol files are, and to control how the symbol attributes
are promoted to the schematic level.  I like to keep this file in the
schematics directory of an individual board.  I also like to keep the
symbols used by the board in the same directory as the
schematics. This directory tree shows how I arrange things:

\begin{center}
  \begin{minipage}{3cm}
    \dirtree{%
      .1 schematics.  .2 gafrc.  .2 jpeda.  .3 symbols.  }
  \end{minipage}
\end{center}

Here's an excerpt from pencils project's \texttt{gafrc} file:

\filesnip{~/projects/pencils/implement/schematics/gafrc}{
  \verbatiminput{gafrc_sample.txt} }

The individual tools can be further customized with individual
\texttt{rc} files like \texttt{gschemrc}, which are then also placed
in the \verb8~./gEDA8 directory.

\subsection{The \texttt{gschemrc} file}
The configuration options available for your local
\verb8~./gEDA/gschemrc8 are shown in the ``system-gschemrc'' file.  On
my machine, this file is located in
\begin{center}
  \begin{minipage}{5cm}
    \verb8/usr/share/gEDA/system-gschemrc8.
  \end{minipage}
\end{center}


\subsection{PCB configuration}
The configuration file for PCB will vary with the GUI toolkit PCB was
compiled with.  Both lesstif and gtk are supported as of
\verb8pcb-200604148.  Use
\begin{center}
  \begin{minipage}{2cm}
    \verb8pcb -h8
  \end{minipage}
\end{center}
and look for ``Available GUI hid:'' to determine what's been installed
and see a list of configuration options.  The double-dash options
correspond to X and gtk resources.  Example:
\begin{center}
  \begin{minipage}{7cm}
    Command line:\\
    \verb8pcb --grid-color '#ff0000'8\\

    .Xdefaults (for lesstif toolkit):\\
    \verb8pcb.grid-color: #ff00008\\

    \$HOME/.pcb/preferences (for gtk toolkit):\\
    \verb8grid-color = #ff00008
  \end{minipage}
\end{center}

% You can also print and export right from the command line.  We use
% this to generate the documentation; the "master" for illustrations
% are .pcb files!  We invoke pcb from the Makefile to build png (for
% html), eps (for dvi), and pdf (for pdf) formats for each board for
% inclusion in the documentation.
