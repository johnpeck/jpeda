\section{gEDA configuration files}

\subsection{The global \texttt{gafrc} file}
The gEDA suite consists of many different tools, which often need to know where your component libraries are.  The \texttt{gafrc} file is a repository for this information.  It can either be placed in \verb8~./gEDA8, or in an individual project directory.  Here's an excerpt from the \texttt{gafrc} file I keep in \verb8~./gEDA8:

\begin{center}
\begin{minipage}{15cm}
\verbatiminput{/home/john/.gEDA/gafrc_sample}
\end{minipage}
\end{center}

The individual tools can be further customized with individual \texttt{rc} files like \texttt{gschemrc}, which are then also placed in the \verb8~./gEDA8 directory.

\subsection{The \texttt{gschemrc} file}
The configuration options available for your local \verb8~./gEDA/gschemrc8 are shown in the ``system-gschemrc'' file.  On my machine, this file is located in
\begin{center}
\begin{minipage}{5cm}
\verb8/usr/share/gEDA/system-gschemrc8.
\end{minipage}
\end{center}


\subsection{PCB configuration}
The configuration file for PCB will vary with the GUI toolkit PCB was compiled with.  Both lesstif and gtk are supported as of \verb8pcb-200604148.  Use
\begin{center}
\begin{minipage}{2cm}
\verb8pcb -h8
\end{minipage}
\end{center}
and look for ``Available GUI hid:'' to determine what's been installed and see a list of configuration options.  The double-dash options correspond to X and gtk resources.  Example:
\begin{center}
\begin{minipage}{7cm}
Command line:\\
\verb8pcb --grid-color '#ff0000'8\\

.Xdefaults (for lesstif toolkit):\\
\verb8pcb.grid-color: #ff00008\\

\$HOME/.pcb/preferences (for gtk toolkit):\\
\verb8grid-color = #ff00008
\end{minipage}
\end{center}

%You can also print and export right from the command line.  We use
%this to generate the documentation; the "master" for illustrations are
%.pcb files!  We invoke pcb from the Makefile to build png (for html),
%eps (for dvi), and pdf (for pdf) formats for each board for inclusion
%in the documentation.
