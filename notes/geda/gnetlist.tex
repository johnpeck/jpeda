%netlist
\section{Working with \texttt{gnetlist}}

\subsection{Generating a netlist}

The command
\begin{center}
	\texttt{gnetlist -g PCB <input.sch> -o <output.net>}
\end{center}
will create a netlist readable by both you and PCB. The file will have three columns: net name, starting pin, and ending pin.


\subsection{Generating a bill of materials (BOM)}
The \texttt{-g} flag for gnetlist allows you to specify a guile script to be run.  Scripts for BOM creation are located in \verb8/usr/share/gEDA/scheme8, and some are described in table \ref{bom_scripts}.
%Set length for width of description column
\newlength{\descwidth}
\setlength{\descwidth}{11cm}

%Set length for width of script column
\newlength{\namewidth}
\setlength{\namewidth}{3cm}

%Set extra height for cells
\newlength{\heightpad}
\setlength{\heightpad}{0.5cm}

\begin{table}[ht]
	\begin{center}
		\begin{tabular}{|c|c|}
			\hline
			
			%The column headings
			\begin{minipage}[c][\height+\heightpad][c]{\namewidth}
    				\begin{center}
    					Script
    				\end{center}
			\end{minipage}		
			
			&\begin{minipage}[c][\height+\heightpad][c]{\descwidth}
    				\begin{center}
    					Description
    				\end{center}
			\end{minipage}\\
			\hline \hline
			
			%bom
			\begin{minipage}[c][\height+\heightpad][c]{\namewidth}
 				\begin{center}
    					\texttt{bom}
    				\end{center}
			\end{minipage}
			&\begin{minipage}[c][\height+\heightpad][c]{\descwidth}
    				Reads an ``attribs'' file and generates a bill of materials with each component on a separate line. Columns are seperated by tab characters. Lines are not sorted.	
			\end{minipage}\\
			\hline
			
			%bom2
			\begin{minipage}[c][\height+\heightpad][c]{\namewidth}
 				\begin{center}
    					\texttt{bom2}
    				\end{center}
			\end{minipage}
			&\begin{minipage}[c][\height+\heightpad][c]{\descwidth}
				Reads an ``attribs'' file and generates a bill of materials grouping components with the same value attribute on the same line.  Columns are separated by colons.  Different items in the same column are seperated by a comma character.  Later versions of this backend have a quantity column appended to whatever appears in \verb8attribs8.  Find the latest version at \verb8git.gpleda.org8.
			\end{minipage}\\
			\hline			
			
			%partslist1
			\begin{minipage}[c][\height+\heightpad][c]{\namewidth}
 				\begin{center}
    					\texttt{partslist1}
    				\end{center}
			\end{minipage}
			&\begin{minipage}[c][\height+\heightpad][c]{\descwidth}
				Creates a bill of materials listing each component on a separate line.  Unlike the \texttt{bom} script, this always produces ``refdes,'' ``device,'' ``value,'' ``footprint,'' and ``quantity'' columns.  No attribute file is required.  Lines are sorted alphabetically by refdes.  Since every line contains just one part, the quantity is always 1.
			\end{minipage}\\
			\hline	
			
			%partslist2
			\begin{minipage}[c][\height+\heightpad][c]{\namewidth}
 				\begin{center}
    					\texttt{partslist2}
    				\end{center}
			\end{minipage}
			&\begin{minipage}[c][\height+\heightpad][c]{\descwidth}
				Similar to \texttt{partslist1}, but lines are sorted by the value of the device attribute and not the refdes.
			\end{minipage}\\
			\hline						
			
			%partslist3
			\begin{minipage}[c][\height+\heightpad][c]{\namewidth}
 				\begin{center}
    					\texttt{partslist3}
    				\end{center}
			\end{minipage}
			&\begin{minipage}[c][\height+\heightpad][c]{\descwidth}
				Groups components with the same value on the same line, as in \texttt{bom2}.  Lines are sorted by the value of the device attribute. The fourth column reports the number of parts in a line. Columns are seperated by the tab character, items by space.
			\end{minipage}\\
			\hline				

		\end{tabular}
		\caption{Scripts for BOM creation\label{bom_scripts}}
	\end{center}
\end{table}



For example, the command
\begin{center}
	\texttt{gnetlist -g bom <input.sch> -o <output.bom>}
\end{center}
will generate a BOM.  You must have a file called ``attribs'' in the directory from which gnetlist is invoked.  This file should just contain a column of attributes you want included in the BOM.  An example is:


\begin{center}
\begin{minipage}{2cm}
\verbatiminput{attribs}
\end{minipage}
\end{center}


\subsection{General notes}

