\section{Project file etiquette}
For consistency:
\begin{itemize}
    \item A project usually consists of more than one schematic file.  The collection of files should be put under version control, so that the filenames themselves never pick up a version number.
    \item \verb8gsch2pcb8 will tack ``.pcb'' on to the string you give it as ``output-name,'' so you don't need to add it.  The generated file should be moved to the layout directory.  There it should be versioned to keep track of changes.  The netlist should not be versioned since it is generated by the schematic.
    \item If a schematic has more than one page, the first page should be a table of contents.
    \item If the ECO changes the BOM, modify it by hand and append \verb8_eco8 to the filename. I won't bother keeping track of multiple rounds of ECOs -- each new ECO should just overwrite the \verb8_eco8 BOM.  If there's a website with the BOM on it, the BOM should be updated to the \verb8_eco8 version.
\end{itemize}

\clearpage
\subsection{Directory structure}
\label{project_structure_subsection}
A nice directory structure looks like this:
\begin{itemize}
    \item \textbf{project} -- The top-level directory.
        \begin{itemize}
            \item \textbf{datasheets} -- Datasheets for parts used in the project.
            \item \textbf{eda} -- Version-controlled eda tools.
                \subitem Check out using \texttt{svn checkout svn://www.jrranalytics.com/eda}
            \item \textbf{proto} -- Version-controlled prototyping tools.
                \subitem Check out using \texttt{svn checkout svn://www.jrranalytics.com/proto}
            \item \textbf{notes} -- Design notes for the project.
                \subitem \textbf{doctools} -- Version-controlled documentation tools.
                    \subsubitem Check out using \texttt{svn checkout svn://www.jrranalytics.com/doctools}
            \item progress.org -- Progress notes to be edited with org-mode.
            \item \textbf{reva} -- Letter-coded revision directory.
                \begin{itemize}
                    \item \textbf{data}
                    \item \textbf{drawings} -- Mechanical drawings for the revision.
                    \item \textbf{schematics} -- Electrical schematic pages
                        \subitem \textbf{kit2} -- Number-coded directory generated with the \textit{kitgen} script.
                    \item \textbf{layout} -- Circuit board layout
                    \item \textbf{pictures}
                \end{itemize}
        \end{itemize}
\end{itemize}

\subsection{Preparing a project for version control}

\boxcmd{\$ mkdir project \\
        \$ mkdir project/datasheets \\
        \$ mkdir project/notes \\
        \$ touch project/progress.org \\
        \$ mkdir project/reva \\
        \$ mkdir project/reva/data \\
        \$ mkdir project/reva/drawings \\
        \$ mkdir project/reva/schematics \\
        \$ mkdir project/reva/layout \\
        \$ mkdir project/reva/pictures \\
        \$ mkdir project/reva/howto \\
        \$ mkdir project/reva/code \\
        \$ svn import ./project svn://www.jrranalytics.com/projects/project -m "Initial import" \\
        \$ mv project project\_nosvn \\
        \$ svn checkout svn://www.jrranalytics.com/projects/project \\
        \$ rm -rf project\_nosvn
}
Now add the subversion externals:
\boxcmd{\$ cd project \\
        \$ svn propedit svn:externals .
}
...which will prompt you to edit a file defining the externals.  Add the lines
\filesnip{svn-prop.tmp}{eda svn://www.jrranalytics.com/eda \\
            eda/notes/geda/doctools svn://www.jrranalytics.com/doctools \\
            eda/notes/avr/doctools svn://www.jrranalytics.com/doctools \\
            notes/doctools svn://www.jrranalytics.com/doctools \\
            reva/howto/doctools svn://www.jrranalytics.com/doctools
}
to any other externals you'd like to define.  Now a simple \texttt{svn update} will bring in the externals.

\rednote{You must create a task tagged with the project name after creating a project.  The top-level task name should be ``\texttt{* TODO Work on} \textit{project name}'', with a link to the project location below as a comment.}

\subsection{Using the \texttt{makeproj.py} script}
The \texttt{makeproj.py} script automates project creation by creating the directories detailed in section \ref{project_structure_subsection}.  To use it, first edit end script to customize your project's name, home directory, and revision letter.  Then the command
\boxcmd{\$ python makeproj.py}
will create the project.

The script can also make a new revision of an existing project.  Just change the revcode while leaving the existing project name and directory.
