\section{Project file etiquette}
For consistency:
\begin{itemize}
\item A project usually consists of more than one schematic file.  The
  collection of files should be put under version control, so that the
  filenames themselves never pick up a version number.
\item \verb8gsch2pcb8 will tack ``.pcb'' on to the string you give it
  as ``output-name,'' so you don't need to add it.  The generated file
  should be moved to the layout directory.  There it should be
  versioned to keep track of changes.  The netlist should not be
  versioned since it is generated by the schematic.
\item If a schematic has more than one page, the first page should be
  a table of contents.
\item If the ECO changes the BOM, modify it by hand and append
  \verb8_eco8 to the filename. I won't bother keeping track of
  multiple rounds of ECOs -- each new ECO should just overwrite the
  \verb8_eco8 BOM.  If there's a website with the BOM on it, the BOM
  should be updated to the \verb8_eco8 version.
\end{itemize}

\clearpage
\subsection{Directory structure}
\label{project_structure_subsection}
A nice directory structure looks like this:
\begin{center}
  \begin{minipage}{10cm}
    \dirtree{%
      .1 project. 
      .2 progress.org.
      .2 datasheets. 
      .2 notes.
      .2 marketing.
      .3 manual.
      .2 implement.
      .3 implementation notes.
      .3 drawings.
      .3 boards.
      .4 boardname.
      .5 revision.
      .6 schematics.
      .7 numbered kit directory.
      .7 jpeda.
      .6 layout.
      .3 cables.
      .4 cablename.
      .5 revision.
      .3 chassis.
      .4 partname.
      .5 revision. 
      .3 data.
      .4 dated data directory.
      .3 photos.
    }
  \end{minipage}
\end{center}

