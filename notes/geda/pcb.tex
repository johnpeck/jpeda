\section{Working with \texttt{PCB}}

%%%%%%%%%%%%%%%%%%%%%%%%%%%%%%%%%%%%%%%%%%%%%%%%%%%%%%%
% Using the command line
%%%%%%%%%%%%%%%%%%%%%%%%%%%%%%%%%%%%%%%%%%%%%%%%%%%%%%%
\subsection{Using the command line}
Use \kbkey{:} to bring up command-line entry, which can be dismissed
with \kbkey{esc}.

%%%%%%%%%%%%%%%%%%%%%%%%%%%%%%%%%%%%%%%%%%%%%%%%%%%%%%%
% Changing the board size
%%%%%%%%%%%%%%%%%%%%%%%%%%%%%%%%%%%%%%%%%%%%%%%%%%%%%%%
\subsection{Changing the board size}
PCB sets the physical board to be a rectangle the size of your working
area by default.  This working area size can be set using
\menuitem{file} $\rightarrow$ \menuitem{preferences} $\rightarrow$
\menuitem{sizes}.  However, the physical board dimensions and shape
can be made different from the working area by drawing on the
``outline'' layer.

To make a custom board outline, use the line tool to draw a 10-mil
wide line on your outline drawing layer.  Make sure this layer isn't
associated with any physical board layers by looking at
\menuitem{file} $\rightarrow$ \menuitem{preferences} $\rightarrow$
\menuitem{layers} $\rightarrow$ \menuitem{groups}.  The outline group
number should be different from every other.  The 10-mil thickness is
a convention, and the center of the outline drawing lines will become
the actual board outline.



%%%%%%%%%%%%%%%%%%%%%%%%%%%%%%%%%%%%%%%%%%%%%%%%%%%%%%%
% Bringing in a netlist
%%%%%%%%%%%%%%%%%%%%%%%%%%%%%%%%%%%%%%%%%%%%%%%%%%%%%%%
\subsection{Bringing in a netlist}
The \texttt{gsch2pcb} program will generate a netlist file for you to
bring into the \texttt{*.pcb} file it also generates.  Bring it in
using \textsl{file} $\rightarrow$ \textsl{load netlist file}.  As an
added feature, \texttt{gsch2pcb} generates a \texttt{*.cmd} file that
will rename footprint pins to coincide with their schematic parts.
This file has to be processed using pcb's command line entry, accessed
with \textsl{window} $\rightarrow$ \textsl{command entry}.  Once
there, enter
\begin{center}
  \begin{minipage}{5cm}
    \verb8ExecuteFile(*.cmd)8
  \end{minipage}
\end{center}
to process the command file.  Finally, hit ``w'' to make rubber bands
show up.

The rat lines are pretty thick by default.  To change them, edit the
``rat-thickness'' setting in whatever configuration file you have.  I
like 1 mil.




%%%%%%%%%%%%%%%%%%%%%%%%%%%%%%%%%%%%%%%%%%%%%%%%%%%%%%%
% Working with layer groupings
%%%%%%%%%%%%%%%%%%%%%%%%%%%%%%%%%%%%%%%%%%%%%%%%%%%%%%%
\subsection{Working with layer groupings}
PCB allows you to associate different logical routing ``layers'' with
``groups.''  The idea is to allow color-coding of different traces
according to their purposes.  All traces belonging to the same layer
will have the same color.  For example, power traces could be
associated with the ''power'' layer and all be colored red.  Each
logical layer must be associated with one unique group, and each group
ultimately becomes the artwork for one of a board's physical layers.
Thus, you can't have all power traces belong to the same layer if they
exist on different sides of the board.  You can have them all be the
same color though, simply by creating ''power-solder'' and
''power-component'' layers with the same colors.

The layers ``solder side'' and ``component side'' are always defined
since all boards have two sides.  You can't draw traces directly on
them though -- you need to associate them and yet another layer with
the same group.  To draw traces on the component side, you have to
create a layer called something like ``component'' and map it to the
same group as ``component side'' using \textsl{file} $\rightarrow$
\textsl{preferences} $\rightarrow$ \textsl{layers}.

The number of groups available will increase as you add logical
layers.  Specifically, the number of groups will be the same as the
number of layers, excluding the predefined solder and component side
layers.  So, you have to have four custom layers defined to make a
four-layer board.


%%%%%%%%%%%%%%%%%%%%%%%%%%%%%%%%%%%%%%%%%%%%%%%%%%%%%%%
% Silkscreen preferences
%%%%%%%%%%%%%%%%%%%%%%%%%%%%%%%%%%%%%%%%%%%%%%%%%%%%%%%
\subsection{Silkscreen preferences}
The default minimum silkscreen text width of 10 mils makes text and
reference designators hard to read.  Setting this to 5 mils is
acceptable for Sierra Proto Express and makes things more readable.
Using \textsl{file} $\rightarrow$ \textsl{preferences} $\rightarrow$
\textsl{sizes} to change the ``minimum silk width'' parameter will
change the width of all current and future silkscreen text.


%%%%%%%%%%%%%%%%%%%%%%%%%%%%%%%%%%%%%%%%%%%%%%%%%%%%%%%
% Adding a ground/power plane
%%%%%%%%%%%%%%%%%%%%%%%%%%%%%%%%%%%%%%%%%%%%%%%%%%%%%%%
\subsection{Adding a ground/power plane}
Planes of copper are added by drawing rectangles.  These planes become
associated with the first nets they're connected to.  To draw a ground
plane, select the layer associated with your ground plane and draw a
rectangle on it.  All pins and pads sharing the same physical layer
will now be isolated from the rectangle by their individual
``clearance'' distances.  To actually connect one and thus connect
your plane to a net, select the thermal tool and then the pin/via to
be connected.  A thermal relief will show up.  Optimizing the rat's
nest now with \textsl{o} will make little rat circles on pads that
want to connect to the plane.

\begin{table}[htb]
  \begin{center}
    \begin{tabular}{|c|c|}\hline
      Key	&Function \\ \hline \hline

      \textsl{j}	
      &\parbox[c][1.5\height][c]{5cm}{Toggle a track's ``clears polygon'' attribute.  This allows you to connect tracks to planes or copper pours.}\\ 
      \hline

      \textsl{k}
      &\parbox[c][1.5\height][c]{5cm}{Increase isolation radii around vias or through-hole pins in a copper polygon.}\\
      \hline

      $<shift>$\textsl{k}
      &\parbox[c][1.5\height][c]{5cm}{The opposite of \textsl{k}}\\
      \hline

      $\backslash$
      &\parbox[c][1.5\height][c]{5cm}{Toggle ``thin draw'' mode for drawing only the ouline of polygons.  You can't see though filled-in polygons to see tracks on the other side of the board.}\\
      \hline

    \end{tabular}
  \end{center}
  \caption{Key bindings important for working with ground planes and copper pours.  Hotkeys modify the object the cursor hovers over.\label{pour_table}}
\end{table}



\begin{itemize}
\item The thermal tool can only be used on pins or vias -- not pads.
  Pads can be connected to planes by drawing a track between the two
  and using \textsl{j}.
\item New lines drawn on a polygon will not be isolated from it unless
  \textsl{settings} $\rightarrow$ \textsl{new lines, arcs clear
    polygons} is checked.
\end{itemize}



% Adding tracks
\clearpage
\subsection{Routing}
\subsubsection{Adding tracks}
Tracks are added using the line tool.  Turning on 135$^\circ$ mode as
described in table \ref{track_table} makes the tracks look prettier.
As with everything else, you must activate the layer you wish to draw
on before you draw anything.

\begin{table}[ht]
  \begin{center}
    \begin{tabular}{|c|c|}\hline
      Key	&Function \\ \hline \hline

      $\cdot$	&\wksentry{7cm}{Toggle between line and track mode}\\ \hline

      /	&\wksentry{7cm}{Toggle between 45$^\circ$ and 135$^\circ$ track modes}\\ \hline
      c &\wksentry{7cm}{Center the layout on the cursor position}\\ \hline
    \end{tabular}
  \end{center}
  \caption{Key bindings important for making tracks with the line tool.\label{track_table}}
\end{table}

%%%%%%%%%%%%%%%%%%%%%%%%%%%%%%%%%%%%%%%%%%%%%%%%%%%%%%%
% Adding vias
%%%%%%%%%%%%%%%%%%%%%%%%%%%%%%%%%%%%%%%%%%%%%%%%%%%%%%%
\clearpage
\subsubsection{Adding vias}
Vias are added using the via tool.  I think the track-to-via routing
flow is in flux with PCB, but what works is dropping a via on top of a
track to associate it with a net.  When switching layers during
routing, drop a via on the last point of a track, then continue the
track from the via on another layer.  Use the number keys to switch
routing layers.  Some keys important to vias are listed in table
\ref{via_keys}.  Note that the ``via size'' is the overall via
diameter.
\begin{table}[htb]
  \begin{center}
    \begin{tabular}{|c|c|}\hline \hline
      Key	&Function \\ \hline

      $<$\textsl{shift}$>$v				&\parbox[c][1.5\height][c]{5cm}{Via size +5mil}\\ \hline
      $<$\textsl{shift}$>$$<$\textsl{ctrl}$>$v	&\parbox[c][1.5\height][c]{5cm}{Via size -5mil}\\ \hline
      $<$\textsl{alt}$>$v				&\parbox[c][1.5\height][c]{5cm}{Via drill +5mil}\\ \hline
      $<$\textsl{shift}$>$$<$\textsl{alt}$>$v		&\parbox[c][1.5\height][c]{5cm}{Via drill -5mil}\\ \hline
      k						&\parbox[c][1.5\height][c]{5cm}{Clearance in polygons +2mil}\\ \hline
      $<$\textsl{shift}$>$k				&\parbox[c][1.5\height][c]{5cm}{Clearance in polygons -2mil}\\ \hline
    \end{tabular}
  \end{center}
  \caption{Key bindings important for making vias with the via tool.\label{via_keys}}
\end{table}

\subsubsection{Reasonable via sizes}
As for via sizes, the important numbers come from the board
fabrication houses.  A 15-mil drill is a good minimum via drill size,
and an overall via size of 30 mils will give an annulus of 7.5 mils --
a typical minimum annulus is 5 mils.  A polygon clearance of 10 mils
is a good minimum size, and 12 mils is safely above that.  For routing
more power, remember that a 15-mil via has a diameter of about 50
mils.  The hole plating thickness is roughly the same as copper
thickness for boards plated with 1 oz/ft$^2$.  So the minimum via size
can handle quite a bit of power.  Table \ref{via_sizes} lists some
good via sizes to start with.
\begin{table}[ht]
  \begin{center}
    \begin{tabular}{|c|c|c|c|c|}\hline
      &Drill (mils)	&Clearance (mils)	&Via size (mils)\\ \hline \hline
      Signal	&15		&12			&30\\ \hline
      Power	&22		&12			&42\\ \hline
    \end{tabular}
  \end{center}
  \caption{Some good via sizes\label{via_sizes}}
\end{table}

\subsubsection{Sloppy via sizes}
These numbers come from expressPCB, which offers some cheaper products
than Sierra Proto Express.
\begin{table}[ht]
  \begin{center}
    \begin{tabular}{|c|c|c|c|c|}\hline
      &Drill (mils)	&Clearance (mils)	&Via size (mils)\\ \hline \hline
      Signal	&20		&12			&40\\ \hline
      Power	&35		&12			&60\\ \hline
    \end{tabular}
  \end{center}
  \caption{Some via sizes good for expressPCB\label{tab:epcb_sizes}}
\end{table}

\subsubsection{Optimizing the netlist}
After adding a track, press \texttt{o} to optimize the rat's nest.
PCB's message log window will then tell you how many rat lines remain.
These rat lines are generated whenever PCB finds a difference between
the netlist and the actual board connectivity.  They aren't always
visible though, and it can be hard to find pins or pads that need to
be routed.  The solution is to search the working file for the word
``rat.''  A rat line element might look like this:
\filesnip{mylayout.pcb}{ Rat[146200 81200 1 118000 41000 0 ``via''] }
...where the first two numbers are the rat line's origin X and Y
coordinates, in centimils.  The fourth and fifth numbers are the
line's end point.

%%%%%%%%%%%%%%%%%%%%%%%%%%%%%%%%%%%%%%%%%%%%%%%%%%%%%%%
% Adding mounting holes
%%%%%%%%%%%%%%%%%%%%%%%%%%%%%%%%%%%%%%%%%%%%%%%%%%%%%%%
\clearpage
\subsection{Adding mounting holes}
Adding mounting holes works just like manually adding any other
element.  Use \textsl{file} $\rightarrow$ \textsl{load element data to
  paste-buffer} and select your mounting hole footprint file.  Then
you can use the mouse to place the holes on the board.
\begin{itemize}
\item 4-40 mounting holes should be surrounded by a 0.25 x 0.25 inch
  keepout area.
\end{itemize}

%%%%%%%%%%%%%%%%%%%%%%%%%%%%%%%%%%%%%%%%%%%%%%%%%%%%%%%
% Part placement
%%%%%%%%%%%%%%%%%%%%%%%%%%%%%%%%%%%%%%%%%%%%%%%%%%%%%%%
\clearpage
\subsection{Part placement notes}

\begin{itemize}
\item For prototyping, a nice ``pitch'' for 1206 resistors and
  capacitors is 110 mils or greater.
\end{itemize}

%%%%%%%%%%%%%%%%%%%%%%%%%%%%%%%%%%%%%%%%%%%%%%%%%%%%%%%
% Checklist for sending board out for fabrication
%%%%%%%%%%%%%%%%%%%%%%%%%%%%%%%%%%%%%%%%%%%%%%%%%%%%%%%
\clearpage
\subsection{Finished board checklist}
\begin{enumerate}

\item Run the design rule checker (DRC) on your layout using \menuitem{connects} $\rightarrow$      \menuitem{design rule checker}.  This uses the sizes specified in \menuitem{file} $\rightarrow$ \menuitem{preferences} $\rightarrow$ sizes.  Good sizes for Sierra Proto Express are shown in table \ref{sierra_drc}.

\end{enumerate}

\begin{table}[htb]
\begin{center}
\begin{tabular}{|c|c|c|}\hline
&&\\
Size	&Value (mils)	&Comment\\
&&\\ \hline \hline

%Minimum copper spacing
\parbox[c][1.5\height][c]{5cm}{Minimum copper spacing}
&6
&\parbox[c][1.5\height][c]{5cm}{Sierra allows 6 mils}\\ \hline

%Minimum copper width
\parbox[c][1.5\height][c]{5cm}{Minimum copper width}
&6
&\parbox[c][1.5\height][c]{5cm}{Sierra lumps this together with spacing as ``minimum trace and space''}\\ \hline

%Minimum touching copper overlap
\parbox[c][1.5\height][c]{5cm}{Minimum touching copper overlap}
&10
&\parbox[c][1.5\height][c]{5cm}{This is how PCB knows that two conductors are touching, and not a constraint imposed by the board house}\\ \hline

%Minimum silk width
\parbox[c][1.5\height][c]{5cm}{Minimum silk width}
&5
&\parbox[c][1.5\height][c]{5cm}{Sierra allows down to 5 mils}\\ \hline

%Minimum drill diameter
\parbox[c][1.5\height][c]{5cm}{Minimum drill diameter}
&15
&\parbox[c][1.5\height][c]{5cm}{Sierra allows 15 mils}\\ \hline

%Minimum annular ring
\parbox[c][1.5\height][c]{5cm}{Minimum annular ring}
&5
&\parbox[c][1.5\height][c]{5cm}{Sierra allows 5 mils}\\ \hline

\end{tabular}
\end{center}
\caption{Good DRC settings for use with Sierra Proto Express\label{sierra_drc}}
\end{table}


%%%%%%%%%%%%%%%%%%%%%%%%%%%%%%%%%%%%%%%%%%%%%%%%%%%%%%%
% Creating Gerber files
%%%%%%%%%%%%%%%%%%%%%%%%%%%%%%%%%%%%%%%%%%%%%%%%%%%%%%%
\clearpage
\subsection{Creating Gerber files}
I usually create a directory called ``submit'' in the layout directory to contain Gerber (RS-274X) files.  Create these files using \menuitem{file} $\rightarrow$ \menuitem{export layout} $\rightarrow$ \menuitem{gerber}.  These can be inspected using the \verb8gerbv8 program (installed separately but part of gEDA).


