\section{Setting up Gentoo's cross-compiling toolchain}

\subsection{Using \textbf{crossdev} to install \textbf{avr-gcc}}
Gentoo provides the \textbf{crossdev} tool to install cross-compiling
and debugging tools for the AVR.  This will create ebuilds not found
in the central gentoo portage repositories, so the first thing we'll
need to do is create a local portage overlay to hold these.  People
that use overlays have their own ways of managing them.  I only need
one overlay, so I followed the recommendation to add the line
\filesnip{/etc/make.conf}{PORTDIR\_OVERLAY = /usr/local/portage}
in in my make.conf file.  I then need to create the directory with
\boxcmd{\# mkdir /usr/local/portage}
to contain the new overlay.  Now I can just
\boxcmd{\# emerge crossdev}
to get the crossdev package.  The command
\boxcmd{\# crossdev --without-headers --target avr}
can then install the binutils, gcc, and libc packages unique to the
avr architecture.  The \texttt{--without-headers} flag is necessary to
keep gentoo from trying to compile avr-libc-headers before avr-gcc is
actually present.  Apparently, avr-gcc doesn't need any of these libc
headers anyway.

After everything finishes, you can look in the cross-avr directory
\rootcmd{ls /usr/local/portage/cross-avr}{
        \textbf{avr-libc}\hspace{0.5cm}\textbf{binutils}
        \hspace{0.5cm}\textbf{gcc}
        \hspace{0.5cm}\textbf{gdb}
        \hspace{0.5cm}\textbf{insight}
        }
to see the subdirectories that were created.  Insight is a graphical
frontend to the gdb debugger.

\subsubsection{Touching up with some symbolic links}
I've had some problems with the compiler or linker not being able to find some files.  For example, when working with the AVR Butterfly, the linker has trouble finding \texttt{avr5.x} and \texttt{crtm169p.o}.  I tried and failed to modify the makefile to add paths to these files, which I found in
\usercmd{locate avr5x}{/usr/lib/binutils/avr/2.21.1/ldscripts/avr5.x}
\usercmd{locate ctrm169p.o}{/usr/avr/lib/avr5/crtm169p.o}
My fix is just to create symbolic links to where the linker can find things.  For example,
\boxcmd{\# ln -s /usr/lib/binutils/avr/2.21.1/ldscripts /usr/avr/lib/ldscripts}
and
\boxcmd{\# ln -s /usr/avr/lib/avr5/crtm169p.o /usr/avr/lib/crtm169p.o}
allowed the linker to finish for AVR Butterfly development.


\subsection{A closer look at \textbf{avr-gcc}}
\label{gcc_subsection}
The \textbf{crossdev} tool installs the \textbf{avr-gcc} compiler
port.  Use
\begin{center}
    \vspace{-\baselineskip}
    \begin{boxedminipage}[t]{\textwidth}
        \texttt{\# avr-gcc --version}\\
        \texttt{avr-gcc (Gentoo 4.4.0 p1.1) 4.4.0}
    \end{boxedminipage}
\end{center}
to see which version was installed.  Of course, newer versions will
support newer targets.  I want to see if my \textsl{at90usb162} part
is supported, so I type
\begin{center}
    \vspace{-\baselineskip}
    \begin{boxedminipage}[t]{\textwidth}
        \texttt{\# avr-gcc --target-help | grep "at90usb162"}\\
        \texttt{attiny167 at90usb82 at90usb162 atmega8
                atmega48 atmega48p atmega88}
    \end{boxedminipage}
\end{center}
to see that it is.

\clearpage
\subsection{Installing \textbf{avrdude} for programming}
\label{avrdude_subsection}
Installing \textbf{avrdude} is as easy as
\begin{center}
    \vspace{-\baselineskip}
    \begin{boxedminipage}[t]{\textwidth}
        \texttt{\# emerge avrdude}
    \end{boxedminipage}
\end{center}
since it's part of the official portage tree.  When the installer
finishes, use
\begin{center}
    \vspace{-\baselineskip}
    \begin{boxedminipage}[t]{\textwidth}
        \texttt{\# avrdude -p?}
    \end{boxedminipage}
\end{center}
to get a list of supported parts.  I installed version 5.5, which
doesn't support the at90usb162 as it was originally released.  I did
find a patch to add support at
\begin{center}
    \vspace{-0.5\baselineskip}
    \verb+http://savannah.nongnu.org+
    \vspace{-0.5\baselineskip}
\end{center}
on a message board for bugs.  The file was called
\begin{center}
    \vspace{-0.5\baselineskip}
    \verb+avrdude_conf_55_support_usb162_usb82.diff+
    \vspace{-0.5\baselineskip}
\end{center}
and I copied it to
\begin{center}
    \vspace{-0.5\baselineskip}
    \verb+/usr/portage/dev-embedded/avrdude+
    \vspace{-0.5\baselineskip}
\end{center}
to patch the original ebuild.

To activate the patch, I first added the ``inherit eutils'' line to
\begin{center}
    \vspace{-\baselineskip}
    \begin{tabular}{|l|} \hline
        \rowcolor[gray]{0.9}
        \begin{minipage}[c]{\textwidth - 2\tabcolsep}
            \textbf{File:}
            /usr/portage/dev-embedded/avrdude/avrdude-5.5.ebuild
        \end{minipage}\\
    ...\\
    LICENSE="GPL-2"\\
    SLOT="0"\\
    KEYWORDS="~arm amd64 ~ppc ~ppc64 x86"\\
    inherit eutils\\
    ...\\
    \hline
    \end{tabular}
\end{center}
near the beginning of the file.  This enables the ebuild to use
\textbf{epatch}.  I then added the ``epatch'' line
\begin{center}
    \vspace{-\baselineskip}
    \begin{tabular}{|l|} \hline
        \rowcolor[gray]{0.9}
        \begin{minipage}[c]{\textwidth - 2\tabcolsep}
            \textbf{File:}
            /usr/portage/dev-embedded/avrdude/avrdude-5.5.ebuild
        \end{minipage}\\
    ...\\
    src\_unpack() \{    \\
        unpack \${A}    \\
        cd "\${S}"      \\
        \# let the build system re-generate these, bug \#120194 \\
        rm -f lexer.c config\_gram.c config\_gram.h \\
        epatch
        /usr/portage/dev-embedded/
        avrdude/avrdude\_conf\_55\_support\_usb162\_usb82.diff  \\
    \}  \\
    ...\\
    \hline
    \end{tabular}
\end{center}
to the same file to tell the ebuild where to find the patch.  I then
issued the command
\begin{center}
    \vspace{-\baselineskip}
    \begin{boxedminipage}[t]{\textwidth}
        \texttt{\# ebuild /usr/portage/dev-embedded/avrdude/avrdude-5.5.ebuild manifest}
    \end{boxedminipage}
\end{center}
to update the ebuild's manifest.  Finally,
\begin{center}
    \vspace{-\baselineskip}
    \begin{boxedminipage}[t]{\textwidth}
        \texttt{\# emerge avrdude}
    \end{boxedminipage}
\end{center}
issued again applied the patch and recompiled the package.  The ebuild
also needs to be able to update the file \texttt{/etc/avrdude.conf}
before the patch is complete.

When everything is done,
\begin{center}
    \vspace{-\baselineskip}
    \begin{boxedminipage}[t]{\textwidth}
        \texttt{\# avrdude -p?} \\
        \texttt{Valid parts are:}\\
        ...\\
        \texttt{usb162 = AT90USB162      [/etc/avrdude.conf:11738]}\\
        ...
    \end{boxedminipage}
\end{center}
indicates success!
