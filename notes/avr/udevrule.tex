\section{Writing a udev rule}
When something programmed to be a USB communications device class
(CDC) device is plugged into the USB port, \textbf{udev}
will work with \textbf{devfs} to position the device's node in linux's
\texttt{/dev} directory\footnotemark{}.  The node name will be
\texttt{ttyACM0} the first time the device is plugged in, but will
increment to \texttt{ttyACM1} or higher as the device is hotplugged.
This makes communicating with the device confusing.  A \textbf{udev}
can simplify things by creating a static symbolic link to this dynamic
node.

\footnotetext{Note that you must have USB modem (CDC ACM) drivers either
built into your kernel or built as a module for the kernel to
recognize your device.  The relevant checkbox in the kernel
configuration menu is at \texttt{Device drivers} $\rightarrow$
\texttt{USB support} $\rightarrow$ \texttt{USB Modem (CDC ACM) support}}

\textbf{Udev} rules live in the \texttt{/etc/udev/rules.d} directory,
with their filenames beginning with numbers.  Rules are assigned using
the number beginning the filename, with lower numbers being assigned
first.  So rules called out in \texttt{10-somerules.rules} will be
assigned before and take precedence over those in
\texttt{20-morerules.rules}.  I have a file called
\texttt{10-jps.rules}, where 10 is lower than any other file number in
the \texttt{rules.d} directory.

Rules themselves must contain a match and an action.  My rule file
contains
\filesnip{/etc/udev/rules.d/10-jps.rules}
{
    \# The CDC demo\\
    KERNEL=="ttyACM*", SYMLINK+="avr\_tty"
}
where commented lines begin with a \#.  This rule tells \textbf{udev}
to assign the symbolic link \texttt{avr\_tty} to anything the kernel
wants to call \texttt{ttyACM*}.  This causes all of \texttt{ttyACM0},
\texttt{ttyACM1}, and so on to be given the same symbolic link in the
\texttt{/dev} directory.

When your rules are written,
\usercmd{udevadm control --reload-rules}{\vspace{-\baselineskip}}
will make them active.

