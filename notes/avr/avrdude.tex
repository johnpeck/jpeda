\section{Using \texttt{avrdude}}


\subsection{Reading the lock byte}
\label{readlock}
The AT90USB162 has one lock byte and three fuse bytes.  The lock byte 
is important because it will prevent flash programming by default.  
The default value is \texttt{00101100}, or \texttt{0x2c}.  The lowest 
two bits prevent flash programming when cleared.  To read the lock byte:
%
\usercmd{avrdude -c dragon\_isp -p at90usb162 -P usb -U 
lock:r:lockbyte:b}{\verbatiminput{lockout.txt}}
%
...and then look at the \texttt{lockbyte} file to see the value.  The 
\texttt{b} parameter tells avrdude to write the file in binary, so the 
file contains:
\usercmd{cat lockbyte}{\texttt{0b101100}}
with the lowest two bits cleared.  This means that writing to flash is 
not allowed.

\clearpage
\subsection{Erasing the part}
The \texttt{-e} flag will instruct avrdude to erase the part.  For 
example:
%
\usercmd{avrdude -c dragon\_isp -p at90usb162 -P usb 
-e}{\verbatiminput{chiperase.txt}}
%
...erases the part's flash and eeprom.  The part's lock byte should 
change from its factory default after an erase to allow writing to 
flash.  To check this, read the lock byte as in section 
\ref{readlock}.  The output file now contains
\usercmd{cat lockbyte}{\texttt{0b111111}}
with the lowest two bits set.  This indicates that no memory lock 
features are enabled.
